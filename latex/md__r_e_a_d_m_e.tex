A\+E\+D\+A repositorium abertum est

Tema   8   –   \+Boleias   \+Partilhadas   (Parte   1) 

Uma empresa deseja explorar o conceito de carpooling e ridesharing  , e pretende criar um sistema para a                                  gestão de uma rede social de partilha de boleias. Poderá haver dois tipos de utilizadores\+: os registados no                               sistema e aqueles que utilizam o sistema ocasionalmente. Entre os utilizadores, também haverá aqueles que                               desejam disponibilizar as suas viaturas, e aqueles que não têm viaturas para partilhar, mas apenas partilham as viagens.\+ 

Quando um utilizador disponibiliza o seu veículo no sistema, deverá indicar o número de lugares disponíveis, e o                           itinerário que realiza como uma lista de pontos de passagem, sendo o primeiro ponto o endereço de origem da                               viagem e o último ponto o endereço de destino. Utilizadores que possam ter interesse em partilhar a viagem toda                           ou trechos das viagens, podem candidatar-\/se aos lugares disponíveis. É possível que fiquem vagos lugares no                               decorrer da viagem, podendo estar disponíveis a quem desejar realizar o trecho ou parte do trecho                                 remanescente.\+ 

Sendo construído em torno do conceito de redes socais, o sistema também privilegia a formação de grupos de                                 partilha de viagem entre pessoas próximas entre si; os utilizadores, ao se registarem no sistema, podem                                 associar-\/se como “buddy” de outros utilizadores – desta forma é possível criar uma rede de relações diretas e                             indiretas entre utilizadores. O serviço de partilha é mantido pelos próprios utilizadores. Os utilizadores que têm                   viatura própria e a partilham no sistema, pagam apenas uma taxa de manutenção; os utilizadores registados sem                             viatura, pagam uma mensalidade fixa de manutenção mais o que realizarem em número de viagens, durante o                                   mês; os utilizadores que utilizam o serviço esporadicamente devem efetuar pagamento em cada viagem.

O sistema, para além de guardar as relações de amizade (“buddies”) dos utilizadores registados, também                               mantém o histórico das viagens realizadas, incluindo o nome do utilizador proprietário da viatura, os pontos de                         origem e destino da viagem, a hora de início e de fim, assim como o dia emque foi realizada. Adicionalmente                               poderá considerar veículos diferentes, nomeadamente veículos ligeiros (5 lugares), vans (de 7 lugares), entre                             outras opções.\+  